\documentclass{article}

\usepackage{graphicx}
\usepackage{hyperref}
\usepackage{minted}

\begin{document}

\title{K8s-like API Server}
\author{Snapp Team}

\maketitle
\tableofcontents

\section{Introduction}
Design and implement a service with Golang programming language to work like a minimal k8s API server.

Successful HTTP call is indicated by an HTTP OK \textit{2XX} status code and the failed call is indicated by a \textit{non-2XX} status code.
So stick with this standard during your development.

\section{Database}

Feel free to use any kind of database you want.

\section{RESTFul API}
\subsection{Authentication}
Use JWT Tokens for authenticating users

\subsection{Users}
This application needs simple user management and authentication so that users can register and receive a token in order to be able to use other endpoints to create URL, update URL, get the stats of the URLs and also get alerts for their URLs.

\subsubsection{Users Endpoints}
\begin{itemize}
  \item Create a new user (public endpoint)
  \item Generate Token for the user (public endpoint)
\end{itemize}

\subsection{Authorization}

There should be a mapping between users and the resource and verbs which a user have access to. For example, user `x` has access to resource `y` with verbs `get,update,delete` in ns `z`. There should be a default `admin` user which only this user can register new rbacs. The rbac itself is not required to have an rbac.

\begin{itemize}
    \item Register (CRUD) a new rbac (auth-reuquired, only admin)
\end{itemize}

\subsection{Container}
Each User must be able to create, update, delete and get container definitions:

\begin{minted}[]{yaml}
apiVersion: v1
kind: Container
metadata:
  labels: map[string]string (optional)
  name: string (required)
  namespace: string (required)
spec:
  name: string (required)
  image: string (required)
  args: []string (optional)
  env: map[string]string (optional)
  ports: []int (optional)
\end{minted}

Validation:
\begin{itemize}
  \item The container spec should be validated.
  \item User access should be checked
\end{itemize}

API:
\begin{itemize}
  \item Create/Update/Delete a new Container (auth-required)
  \item List Container by namespace and user (auth-required)
\end{itemize}

\subsection{Quota}

Each user have a quota for container count and port count, and user can not create more containers than the quota, or request more total count ports in all created containers above the limit.

The quota for each user can be set only by `admin` user and it does not require any rbac.

\begin{itemize}
  \item Set quota for user (auth-required, admin only)
\end{itemize}


Notice: The public endpoints are accessible by everyone and auth-required endpoints need the token for authentication.

\section{Instructions}

Please create a GitHub repository in \href{https://github.com/orgs/snappcloud-learning/repositories}{snappcloud-learning repo}  and share it with us as soon as you started working on this assignment.

\begin{itemize}
  \item Tell us about the Trade-Offs and the important decisions which you made during the design and development.
  \item Provide docker-compose file so we can easily run and see how your application works
  \item Use concise and good commit messages
  \item Design the API with RESTful standards
  \item We don’t need 100\% test coverage but do your best to write some tests for important parts of the application
  \item Write some documentation about your service in a file called README.md
\end{itemize}

Good Luck!

\vspace{1cm}
\includegraphics[width=.25\textwidth]{./snapp.png}

\end{document}
